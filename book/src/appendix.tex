\appendix
\chapter{Installing \LaTeX}
\begin{intro}
Knuth published the source to \TeX{} back in a time when nobody knew
about Open Source and/or Free Software. The License that comes with \TeX{}
lets you do whatever you want with the source, but you can only call the
result of your work \TeX{} if the program passes a set of tests Knuth has
also provided. This has lead to a situation where we have free \TeX{}
implementations for almost every Operating System under the sun. This chapter
will give some hints on what to install on Linux, macOS and Windows, to
get a working \TeX{} setup.
\end{intro}

\section{What to Install}

To use \LaTeX{} on any computer system, you need several programs.

\begin{enumerate}

\item The \TeX{}/\LaTeX{} program for processing your \LaTeX{} source files
into typeset PDF or DVI documents.

\item A text editor for editing your \LaTeX{} source files. Some products even let
you start the \LaTeX{} program from within the editor.

\item A PDF/DVI viewer program for previewing and printing your
documents.

\item A program to handle \PSi{} files and images for inclusion into
your documents.

\end{enumerate}

For every platforms there are several programs that fit the requirements above.
Here we just tell about the ones we know, like and have some experience
with.

\section{Cross Platform Editor}
\label{sec:texmaker}

While \TeX{} is available on many different computing platforms, \LaTeX{}
editors have long been highly platform specific.

Over the past few years I have come to like Texmaker quite a lot.
Apart from being very a useful editor with integrated pdf-preview and syntax
high-lighting, it has the advantage of running on Windows, Mac and
Unix/Linux equally well.  See \url{http://www.xm1math.net/texmaker} for
further information.  There is also a forked version of Texmaker called
TeXstudio on \url{http://texstudio.sourceforge.net/}.  It also seems well
maintained and is also available for all three major platforms.

You will find some platform specific editor suggestions in the OS sections
below.

\section{\TeX{} on macOS}

\subsection{\TeX{} Distribution}

Just download \wi{MacTeX}. It is a
pre-compiled \LaTeX{} distribution for macOS. \wi{MacTeX} provides a full \LaTeX{}
installation plus a number of additional tools. Get Mac\TeX{} from
\url{http://www.tug.org/mactex/}.

\subsection{macOS \TeX{} Editor}

If you are not happy with our cross-platform suggestion Texmaker (section \ref{sec:texmaker}).

The most popular open source editor for \LaTeX{} on the mac seems to be
\TeX{}shop.  Get a copy from \url{http://www.uoregon.edu/~koch/texshop}. It
is also contained in the \wi{MacTeX} distribution.

Recent \TeX Live distributions contain the \TeX{}works editor
\url{http://texworks.org/} which is a multi-platform editor based on the \TeX{}Shop
design. Since \TeX{}works uses the Qt toolkit, it is available on any platform
supported by this toolkit (macOS, Windows, Linux).

\subsection{Treat yourself to \wi{PDFView}}

Use PDFView for viewing PDF files generated by \LaTeX{}, it integrates tightly
with your \LaTeX{} text editor. PDFView is an open-source application, available from the PDFView website on\\
\url{http://pdfview.sourceforge.net/}. After installing, open
PDFViews preferences dialog and make sure that the \emph{automatically reload
documents} option is enabled and that PDFSync support is set appropriately.

\section{\TeX{} on Windows}

\subsection{Getting \TeX{}}

First, get a copy of the excellent MiK\TeX\index{MiKTeX@MiK\TeX} distribution from\\
\url{http://www.miktex.org/}. It contains all the basic programs and files
required to compile \LaTeX{} documents.  The coolest feature in my eyes, is
that MiK\TeX{} will download missing \LaTeX{} packages on the fly and install them
magically while compiling a document. Alternatively you can also use
the TeXlive distribution which exists for Windows, Unix and Mac OS to
get your base setup going \url{http://www.tug.org/texlive/}.

\subsection{A \LaTeX{} editor}

If you are not happy with our cross-platform suggestion Texmaker (section \ref{sec:texmaker}).

\wi{TeXnicCenter} uses many concepts from the programming-world to provide a nice and
efficient \LaTeX{} writing environment in Windows. Get your copy from\\
\url{http://www.texniccenter.org/}. TeXnicCenter integrates nicely with
MiKTeX.

Recent \TeX Live distributions contain the \TeX{}works Editor
\url{http://texworks.org/}. It supports Unicode and requires at least Windows XP.

\subsection{Document Preview}

You will most likely be using Yap for DVI preview as it gets installed with
MikTeX. For PDF you may want to look at Sumatra
PDF \url{http://blog.kowalczyk.info/software/sumatrapdf/}. I mention Sumatra PDF
because it lets you jump from any position in the pdf document back into
corresponding position in your source document.

\subsection{Working with graphics}

Working with high quality graphics in \LaTeX{} means that you have to use
\EPSi{} (eps) or PDF as your picture format. The program that helps you
deal with this is called \wi{GhostScript}. You can get it, together with its
own front-end \wi{GhostView}, from \url{http://www.cs.wisc.edu/~ghost/}.

If you deal with bitmap graphics (photos and scanned material), you may want
to have a look at the open source Photoshop alternative \wi{Gimp}, available
from \url{http://gimp-win.sourceforge.net/}.

\section{\TeX{} on Linux}

If you work with Linux, chances are high that \LaTeX{} is already installed
on your system, or at least available on the installation source you used to
setup. Use your package manager to install the following packages:

\begin{itemize}
\item texlive -- the base \TeX{}/\LaTeX{} setup.
\item emacs (with AUCTeX) -- an editor that integrates tightly with \LaTeX{} through the add-on AUCTeX package.
\item ghostscript -- a \PSi{} preview program.
\item xpdf and acrobat -- a PDF preview program.
\item imagemagick -- a free program for converting bitmap images.
\item gimp -- a free Photoshop look-a-like.
\item inkscape -- a free illustrator/corel draw look-a-like.
\end{itemize}

If you are looking for a more windows like graphical editing environment,
check out Texmaker. See section \ref{sec:texmaker}.

Most Linux distros insist on splitting up their \TeX{} environments into a
large number of optional packages, so if something is missing after your
first install, go check again.

\chapter{Notes on Babel Language Support}

\begin{intro}
When writing text in a non-english language, you may have come across the \pai{bable} package. These days it has been obsoleted by the \pai{polyglossia} package, but if you like to read up on history, here are some notes of typesetting in different languages.
\end{intro}

\subsection{Support for French}

\secby{Daniel Flipo}{daniel.flipo@univ-lille1.fr}
Some hints for those creating \wi{French} documents with \LaTeX{}:
load French language support with the following command:

\begin{lscommand}
\verb|\usepackage[francais]{babel}|
\end{lscommand}

This enables French hyphenation, if you have configured your
\LaTeX{} system accordingly. It also changes all automatic text into
French: \verb+\chapter+ prints Chapitre, \verb+\today+ prints the current
date in French and so on. A set of new commands also
becomes available, which allows you to write French input files more
easily. Check out table \ref{cmd-french} for inspiration.

\begin{table}[!htbp]
\caption{Special commands for French.} \label{cmd-french}
\begin{lined}{9cm}
\selectlanguage{french}
\begin{tabular}{ll}
\verb+\og guillemets \fg{}+         \quad &\og guillemets \fg \\[1ex]
\verb+M\up{me}, D\up{r}+            \quad &M\up{me}, D\up{r}  \\[1ex]
\verb+1\ier{}, 1\iere{}, 1\ieres{}+ \quad &1\ier{}, 1\iere{}, 1\ieres{}\\[1ex]
\verb+2\ieme{} 4\iemes{}+           \quad &2\ieme{} 4\iemes{}\\[1ex]
\verb+\No 1, \no 2+                 \quad &\No 1, \no 2   \\[1ex]
\verb+20~\degres C, 45\degres+      \quad &20~\degres C, 45\degres \\[1ex]
\verb+\bsc{M. Durand}+              \quad &\bsc{M.~Durand} \\[1ex]
\verb+\nombre{1234,56789}+          \quad &\nombre{1234,56789}
\end{tabular}
\selectlanguage{english}
\bigskip
\end{lined}
\end{table}

You will also notice that the layout of lists changes when switching to the
French language. For more information on what the \texttt{francais}
option of \textsf{babel} does and how to customize its behaviour, run
\LaTeX{} on file \texttt{frenchb.dtx} and read the produced file
\texttt{frenchb.dvi}.

Recent versions of \pai{frenchb} rely on \pai{numprint} to implement the \ci{nombre} command.

\subsection{Support for German}

Some hints for those creating \wi{German}\index{Deutsch}
documents with \LaTeX{}: load German language support with the following
command:

\begin{lscommand}
\verb|\usepackage[german]{babel}|
\end{lscommand}

This enables German hyphenation, if you have configured your
\LaTeX{} system accordingly. It also changes all automatic text into
German. Eg. ``Chapter'' becomes ``Kapitel.'' A set of new commands also
becomes available, which allows you to write German input files more quickly
even when you don't use the inputenc package. Check out table
\ref{german} for inspiration. With inputenc, all this becomes moot, but your
text also is locked in a particular encoding world.

\begin{table}[!htbp]
\caption{German Special Characters.} \label{german}
\begin{lined}{8cm}
\selectlanguage{german}
\begin{tabular}{*2{ll}}
\verb|"a| & "a \hspace*{1ex} & \verb|"s| & "s \\[1ex]
\verb|"`| & "` & \verb|"'| & "' \\[1ex]
\verb|"<| or \ci{flqq} & "<  & \verb|">| or \ci{frqq} & "> \\[1ex]
\ci{flq} & \flq & \ci{frq} & \frq \\[1ex]
\ci{dq} & " \\
\end{tabular}
\selectlanguage{english}
\bigskip
\end{lined}
\end{table}

In German books you often find French quotation marks (\flqq guil\-le\-mets\frqq).
German typesetters, however, use them differently. A quote in a German book
would look like \frqq this\flqq. In the German speaking part of Switzerland,
typesetters use \flqq guillemets\frqq~the same way the French do.

A major problem arises from the use of commands
like \verb+\flq+: If you use the OT1 font (which is the default font) the
guillemets will look like the math symbol ``$\ll$'', which turns a typesetter's stomach.
T1 encoded fonts, on the other hand, do contain the required symbols. So if you are using this type
of quote, make sure you use the T1 encoding. (\verb|\usepackage[T1]{fontenc}|)

\subsection[Support for Korean]{Support for Korean\footnotemark}\label{support_korean}%
\footnotetext{Written by Karnes Kim $<$\href{mailto:karnes@ktug.org}{karnes@ktug.org}$>$ and Kihwang Lee $<$\href{mailto:leekh@ktug.org}{leekh@ktug.org}$>$ on behalf of the Korean \TeX{} Users Group and the Korean \TeX{} Society.}

To process Hangul\footnote{Hangul is the name of the Korean writing system. Refer to \url{http://en.wikipedia.org/wiki/Hangul} for more information.} characters or prepare a document written in Korean using \LaTeX, put the following code in the preamble of the document.

\begin{lscommand}
\verb|\usepackage{kotex}|
\end{lscommand}

A document containing the declaration above will have to be processed by
pdf\LaTeX, \hologo{XeLaTeX}, or Lua\LaTeX{}.  Make sure that the input file
written in Hangul is encoded in Unicode UTF-8.  The package called
ko.\TeX\footnote{Reads ``Korean \TeX{}''. ko.\TeX{} is the name of a
collection of packages including \texttt{cjk-ko}, \texttt{kotex-utf},
\texttt{xetexko}, and \texttt{luatexko}.} is under continuous development by
the Korean \TeX{} Users Group\footnote{\url{http://ktug.org}} and the Korean
\TeX{} Society.\footnote{\url{http://ktug.kr}} Many people use this package
to create Korean documents for their everyday needs.  ko.\TeX\ has been
available on CTAN since 2014.  It is included \TeX\,Live, MiK\TeX{} and
other modern \TeX{} distributions.  So in all likelihood you can start
working right away without installing any extra packages.

ko.\TeX\ does not use the \texttt{babel} package. Many functions related to
Korean can be activated using the options and configuration commands
provided by the \pai{kotex} package.  If you want to compose a real world
Korean document, you are advised to consult the package documentation (These
documents are written in Korean).

With ko.\TeX, you also get \pai{oblivoir}, a \pai{memoir}-based document
class, tailored for Korean document preparation. So your Korean document
would start like this:

\begin{lscommand}
\verb|\documentclass{oblivoir}|
\end{lscommand}

To generate an index for a Korean document, execute \texttt{komkindex}
instead of \texttt{makeindex}.  It is a version of the \texttt{makeindex}
utility modified for Korean processing.  For lexicographical sorting of the
Korean index items, you can use index style \texttt{kotex.ist} provided by
ko.\TeX{} as follows:

\begin{lscommand}
\verb|komkindex -s kotex foo.idx|
\end{lscommand}

You can also use \texttt{xindy} for index generation as the Korean module for \texttt{xindy} is included in \TeX\,Live.

There is another Korean/Hangul typesetting package called CJK. As the name of the package suggests, it has facilities for typesetting Chinese, Japanese, and Korean characters. It supports multiple encodings of the CJK characters. The following is a simple example of typesetting UTF-8 encoded Hangul using CJK package. It is useful when you submit a manuscript to some academic journals that allow typesetting author names in native languages.

\begin{verbatim}
\usepackage{CJK}

\begin{CJK}{UTF8}{}
\CJKfamily{nanummj}
...
\end{CJK}
\end{verbatim}

\subsection{Writing in Greek}
\secby{Nikolaos Pothitos}{pothitos@di.uoa.gr}
See table~\ref{preamble-greek} for the preamble you need to write in the
\wi{Greek} \index{Greek} language.  This preamble enables hyphenation and
changes all automatic text to Greek.

\begin{table}[btp]
\caption{Preamble for Greek documents.} \label{preamble-greek}
\begin{lined}{7cm}
\begin{verbatim}
\usepackage[english,greek]{babel}
\end{verbatim}
\bigskip
\end{lined}
\end{table}

A set of new commands also becomes available, which allows you to write
Greek input files more easily.  In order to temporarily switch to English
and vice versa, one can use the commands \verb|\textlatin{|\emph{english
text}\verb|}| and \verb|\textgreek{|\emph{greek text}\verb|}| that both take
one argument which is then typeset using the requested font encoding.
Otherwise use the command \verb|\selectlanguage{...}| described in a
previous section.  Check out table~\ref{sym-greek} for some Greek
punctuation characters.  Use \verb|\euro| for the Euro symbol.

\begin{table}[!htbp]
\caption{Greek Special Characters.} \label{sym-greek}
\begin{lined}{4cm}
\selectlanguage{french}
\begin{tabular}{*2{ll}}
\verb|;| \hspace*{1ex}  &  $\cdot$ \hspace*{1ex}  &  \verb|?| \hspace*{1ex}&  ;   \\[1ex]
\verb|((|               &  \og                    &  \verb|))|&  \fg \\[1ex]
\verb|``|               &  `                      &  \verb|''| &  '   \\
\end{tabular}
\selectlanguage{english}
\bigskip
\end{lined}
\end{table}


\subsection{Support for Cyrillic}

\secby{Maksym Polyakov}{polyama@myrealbox.com}
Version~3.7h of \pai{babel} includes support for the
\pai{T2*}~encodings and for typesetting Bulgarian, Russian and
Ukrainian texts using Cyrillic letters.

Support for Cyrillic is based on standard \LaTeX{} mechanisms plus
the \pai{fontenc} and \pai{inputenc} packages. But, if you are going to
use Cyrillics in math mode, you need to load \pai{mathtext} package
before \pai{fontenc}:\footnote{If you use \AmS-\LaTeX{} packages,
load them before \pai{fontenc} and \pai{babel} as well.}
\begin{lscommand}
\verb+\usepackage{mathtext}+\\
\verb+\usepackage[+\pai{T1}\verb+,+\pai{T2A}\verb+]{fontenc}+\\
\verb+\usepackage[english,bulgarian,russian,ukranian]{babel}+
\end{lscommand}

Generally, \pai{babel} will automatically choose the default font encoding,
for the above three languages this is \pai{T2A}.  However, documents are not
restricted to a single font encoding. For multi-lingual documents using
Cyrillic and Latin-based languages it makes sense to include Latin font
encoding explicitly. \pai{babel} will take care of switching to the appropriate
font encoding when a different language is selected within the document.

In addition to enabling hyphenations, translating automatically
generated text strings, and activating some language specific
typographic rules (like \ci{frenchspacing}), \pai{babel} provides some
commands allowing typesetting according to the standards of
Bulgarian, Russian, or Ukrainian languages.


For all three languages, language specific punctuation is provided:
The Cyrillic dash for the text (it is little narrower than Latin dash and
surrounded by tiny spaces), a dash for direct speech, quotes, and
commands to facilitate hyphenation, see Table~\ref{Cyrillic}.

% Table borrowed from Ukrainian.dtx
\begin{table}[htb]
  \begin{center}
  \index{""-@\texttt{""}\texttt{-}}
  \index{""---@\texttt{""}\texttt{-}\texttt{-}\texttt{-}}
  \index{""=@\texttt{""}\texttt{=}}
  \index{""`@\texttt{""}\texttt{`}}
  \index{""'@\texttt{""}\texttt{'}}
  \index{"">@\texttt{""}\texttt{>}}
  \index{""<@\texttt{""}\texttt{<}}
  \caption[Bulgarian, Russian, and Ukrainian]{The extra definitions made
           by Bulgarian, Russian, and Ukrainian options of \pai{babel}}\label{Cyrillic}
  \begin{tabular}{@{}p{.1\hsize}@{}p{.9\hsize}@{}}
   \hline
   \verb="|= & disable ligature at this position.               \\
   \verb|"-| & an explicit hyphen sign, allowing hyphenation
               in the rest of the word.                         \\
   \verb|"---| & Cyrillic emdash in plain text.                      \\
   \verb|"--~| & Cyrillic emdash in compound names (surnames).       \\
   \verb|"--*| & Cyrillic emdash for denoting direct speech.         \\
   \verb|""| & like \verb|"-|, but producing no hyphen sign
               (for compound words with hyphen, e.g.\ \verb|x-""y|
               or some other signs  as ``disable/enable'').     \\
   \verb|"~| & for a compound word mark without a breakpoint.        \\
   \verb|"=| & for a compound word mark with a breakpoint, allowing
          hyphenation in the composing words.                   \\
   \verb|",| & thinspace for initials with a breakpoint
           in following surname.                                \\
   \verb|"`| & for German left double quotes
               (looks like ,\kern-0.08em,).                     \\
   \verb|"'| & for German right double quotes (looks like ``).       \\%''
   \verb|"<| & for French left double quotes (looks like $<\!\!<$).  \\
   \verb|">| & for French right double quotes (looks like $>\!\!>$). \\
   \hline
  \end{tabular}
  \end{center}
\end{table}


The Russian and Ukrainian options of \pai{babel} define the commands \ci{Asbuk}
and \ci{asbuk}, which act like \ci{Alph} and \ci{alph}\footnote{the commands for turning counters into a, b, c, \ldots}, but produce capital
and small letters of Russian or Ukrainian alphabets (whichever is the
active language of the document). The Bulgarian option of \pai{babel}
provides the commands \ci{enumBul} and \ci{enumLat} (\ci{enumEng}), which
make \ci{Alph} and \ci{alph} produce letters of either
Bulgarian or Latin (English) alphabets. The default behaviour of
\ci{Alph}  and \ci{alph} for the Bulgarian language option is to
produce letters from the Bulgarian alphabet.

%Finally, math alphabets are redefined and  as well as the commands for math
%operators according to Cyrillic typesetting traditions.

\subsection{Support for Mongolian}

To use \LaTeX{} for typesetting Mongolian you have a choice between two packages:
Multilingual Babel and Mon\TeX{} by Oliver Corff.

Mon\TeX{} includes support for both Cyrillic and traditional
Mongolian Script. In order to access the commands of Mon\TeX{}, add:
\begin{lscommand}
\ci{usepackage}\verb|[|\emph{language},\emph{encoding}\verb|]{mls}|
\end{lscommand}
\noindent to the preamble. Choose the \emph{language} option \pai{xalx} to generate
captions and dates in Modern Mongolian. To write a complete document in the traditional Mongolian script
you have to choose \pai{bicig} for the \emph{language} option. The document
language option \pai{bicig} enables the ``Simplified Transliteration'' input method.

Enable and disable Latin Transliteration Mode with
\begin{lscommand}
\verb|\SetDocumentEncodingLMC|
\end{lscommand}
and
\begin{lscommand}
\verb|\SetDocumentEncodingNeutral|
\end{lscommand}

More information about Mon\TeX{} is available from
\CTAN|language/mongolian/montex/doc|.

Mongolian Cyrillic script is supported by \pai{babel}. Activate
Mongolian language support with the following commands:

\begin{lscommand}
\verb|\usepackage[mongolian]{babel}|
\end{lscommand}
