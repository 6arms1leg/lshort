\appendix
\chapter{Installing \LaTeX}
\begin{intro}
Knuth published the source to \TeX{} back in a time when nobody knew
about OpenSource and/or Free Software. The License that comes with \TeX{}
lets you do whatever you want with the source, but you can only call the
result of your work \TeX{} if the program passes a set of tests Knuth has
also provided. This has lead to a situation where we have free \TeX{}
implementations for almost every Operating System under the Sun. In this chapter
you will give some hints on what to install on Linux, Mac OS X and Windows to
get \TeX{} working.
\end{intro}

\section{What to Install}

For using LaTeX on any computer system, you need 3 essential pieces of
software:

\begin{enumerate}

\item A text editor for editing your LaTeX source files.

\item The \TeX{}/\LaTeX{} program for processing your \LaTeX{} source files
into typeset PDF or DVI documents.

\item A PDF/DVI viewer program for previewing and printing your
documents.

\item A program to handle PostScript files and images for inclusion into
your documents.

\end{enumerate}

For all platforms there are many programs that fit the requirements above.
Here we just tell about the ones we know, like and have some experience
with.

\section{\TeX{} on Mac OS X}

\subsection{Picking an Editor}

Base your LaTeX environment on the \wi{TextMate} editor! TextMate is not
only a highly customizable, general purpose text editor, it also provides
excellent LaTeX support and integrates tightly with the PDFView previewer.
This combination of tools, lets you use LaTeX in a convenient and Mac-like
manner. You can download a free trial version from the Textmate website on
\url{http://macromates.com/} and purchase a full version for 39 EUR. If you
know an equivalent OpenSource tool for the Mac, please let us know.

\subsection{Get a \TeX{} Distribution}

If you are already using Macports or Fink for installing Unix software under
OS X, install LaTeX using these package managers. Macport users install
LaTeX with \framebox{\texttt{port install tetex}},
Fink users use the command \framebox{\texttt{fink install tetex}}.

If you are neither using Macports nor Fink, download \wi{MacTeX}, which is a
precompiled LaTeX distribution for OS X. \wi{MacTeX} provides a full LaTeX
installation plus a number of additional tools. Get MaxTeX from
\url{http://www.tug.org/mactex/}.

\subsection{Treat yourself to \wi{PDFView}}

Use PDFView for viewing PDF files generated by LaTeX, it integrates tightly
with your LaTeX text editor. PDFView is an open-source application can be
downloaded from the PDFView website on\\
\url{http://pdfview.sourceforge.net/}. Download and install PDFView. Open
PDFViews preferences dialog and make sure that the \emph{automatically reload
documents} option is enabled and that PDFSync support is set to the TextMate
preset.

\section{\TeX{} on Windows}

\subsection{Getting \TeX{}}

First, get a copy of the excellent \wi{MiKTeX} distribution from\\
\url{http://www.miktex.org/}. It contains all the basic programs and files
required to compile \LaTeX{} documents.  The coolest feature in my eyes, is
that MiKTeX will download missing \LaTeX{} packages on the fly and install them
magically while compiling a document.

\subsection{A \LaTeX{} editor}

\LaTeX{} is a programming language for text documents. \wi{TeXnicCenter}
uses many concepts from the programming-world to provide a nice and
efficient \LaTeX{} writing environment in Windows. Get your copy from\\
\url{http://www.toolscenter.org}. TeXnicCenter integrates nicely with
MiKTeX.

An another excellent choice is the editor provided by the LEd project available on
\url{http://www.latexeditor.org}.

\subsection{Working with graphics}

Working with high quality graphics in \LaTeX{} means that you have to use
Postscript (eps) or PDF as your picture format. The program that helps you
deal with this is called \wi{GhostScript}. You can get it, together with its
own front-end \wi{GhostView}, from \url{http://www.cs.wisc.edu/~ghost/}.

If you deal with bitmap graphics (photos and scanned material), you may want
to have a look at the open source photoshop alternative \wi{Gimp} available
from \url{http://gimp-win.sourceforge.net/}.

\section{\TeX{} on Linux}

If you work with Linux, chances are high that \LaTeX{} is already installed
on your system, or at least available on the installation source you used to
setup. Use your package manager to install the following packages:

\begin{itemize}
\item tetex or texlive -- the base \TeX{}/\LaTeX{} setup.
\item emacs (with auctex) -- a Linux editor that integrates tightly with \LaTeX{} through the add-on AucTeX package.
\item ghostscript -- a PostScript preview program.
\item xpdf and acrobat -- a PDF preview program.
\item imagemagick -- a free program for converting bitmap images.
\item gimp -- a free photoshop look-a-like.
\item inkscape -- a free illustrator/corel draw look-a-like.
\end{itemize}

Most Linux distros insist on splitting up their \TeX{} environments into a
large number of optional packages, so if something is missing after your
first install, go check again.
