%%%%%%%%%%%%%%%%%%%%%%%%%%%%%%%%%%%%%%%%%%%%%%%%%%%%%%%%%%%%%%%%%
% Contents: Customising LaTeX output
% $Id$
%%%%%%%%%%%%%%%%%%%%%%%%%%%%%%%%%%%%%%%%%%%%%%%%%%%%%%%%%%%%%%%%%
\chapter{Customising \LaTeX}

\begin{intro}
Documents produced with the commands you have learned up to this
point will look acceptable to a large audience. While they are not
fancy-looking, they obey all the established rules of good
typesetting, which will make them easy to read and pleasant to look at.

However, there are situations where \LaTeX{} does not provide a
command or environment that matches your needs, or the output
produced by some existing command may not meet your requirements.

In this chapter, I will try to give some hints on
how to teach \LaTeX{} new tricks and how to make it produce output
that looks different from what is provided by default.
\end{intro}


\section{New Commands, Environments and Packages}

You may have noticed that all the commands I introduce in this
book are typeset in a box, and that they show up in the index at the end
of the book. Instead of directly using the necessary \LaTeX{} commands
to achieve this, I have created a \wi{package} in which I defined new
commands and environments for this purpose. Now I can simply write:

\begin{example}
\begin{lscommand}
\ci{dum}
\end{lscommand}
\end{example}

In this example, I am using both a new environment called\\
\ei{lscommand}, which is responsible for drawing the box around the
command, and a new command named \ci{ci}, which typesets the command
name and makes a corresponding entry in the index. Check
this out by looking up the \ci{dum} command in the index at the back
of this book, where you'll find an entry for \ci{dum}, pointing to
every page where I mentioned the \ci{dum} command.

If I ever decide that I do not like having the commands typeset in
a box any more, I can simply change the definition of the
\texttt{lscommand} environment to create a new look. This is much
easier than going through the whole document to hunt down all the
places where I have used some generic \LaTeX{} commands to draw a
box around some word.


\subsection{New Commands}

To add your own commands, use the
\begin{lscommand}
\ci{newcommand}\verb|{|%
       \emph{name}\verb|}[|\emph{num}\verb|]{|\emph{definition}\verb|}|
\end{lscommand}
\noindent command.
Basically, the command requires two arguments: the \emph{name} of the
command you want to create, and the \emph{definition} of the command.
The \emph{num} argument in square brackets is optional and specifies the number
of arguments the new command takes (up to 9 are possible).
If missing it defaults to 0, i.e. no argument allowed.

The following two examples should help you to get the idea.
The first example defines a new command called \ci{tnss}. This is
short for ``The Not So Short Introduction to \LaTeXe.'' Such a command
could come in handy if you had to write the title of this book over
and over again.

\begin{example}
\newcommand{\tnss}{The not
    so Short Introduction to
    \LaTeXe}
This is ``\tnss'' \ldots{}
``\tnss''
\end{example}

The next example illustrates how to define a new
command that takes two arguments.
The \verb|#1| tag gets replaced by the first argument you specify,
\verb|#2| with the second argument, and so on.

\begin{example}
\newcommand{\txsit}[2]
 {This is the \emph{#1}
  #2 Introduction to \LaTeXe}
% in the document body:
\begin{itemize}
\item \txsit{not so}{short}
\item \txsit{very}{long}
\end{itemize}
\end{example}

\LaTeX{} will not allow you to create a new command that would
overwrite an existing one. But there is a special command in case you
explicitly want this: \ci{renewcommand}.
It uses the same syntax as the \verb|\newcommand|
command.

In certain cases you might also want to use the \ci{providecommand}
command. It works like \ci{newcommand}, but if the command is
already defined, \LaTeXe{} will silently ignore it.

There are some points to note about whitespace following \LaTeX{} commands. See
page \pageref{whitespace} for more information.

\subsection{New Environments}
Just as with the \verb|\newcommand| command, there is a command
to create your own environments. The \ci{newenvironment} command uses the
following syntax:

\begin{lscommand}
\ci{newenvironment}\verb|{|%
       \emph{name}\verb|}[|\emph{num}\verb|]{|%
       \emph{before}\verb|}{|\emph{after}\verb|}|
\end{lscommand}

Again \ci{newenvironment} can have
an optional argument. The material specified
in the \emph{before} argument is processed before the text in the
environment gets processed. The material in the \emph{after} argument gets
processed when the \verb|\end{|\emph{name}\verb|}| command is encountered.

The example below illustrates the usage of the \ci{newenvironment}
command.
\begin{example}
\newenvironment{king}
 {\rule{1ex}{1ex}%
      \hspace{\stretch{1}}}
 {\hspace{\stretch{1}}%
      \rule{1ex}{1ex}}

\begin{king}
My humble subjects \ldots
\end{king}
\end{example}

The \emph{num} argument is used the same way as in the
\verb|\newcommand| command. \LaTeX{} makes sure that you do not define
an environment that already exists. If you ever want to change an
existing command, use the \ci{renewenvironment} command. It
uses the same syntax as the \ci{newenvironment} command.

The commands used in this example will be explained later. For the
\ci{rule} command see page \pageref{sec:rule}, for \ci{stretch} go to
page \pageref{cmd:stretch}, and more information on \ci{hspace} can be
found on page \pageref{sec:hspace}.

\subsection{Extra Space}

When creating a new environment you may easily get bitten by extra spaces
creeping in, which can potentially have fatal effects, for example when you
want to create a title environment which supresses its own indentation as
well as the one on the following paragraph. The \ci{ignorespaces} command in
the begin block of the environment will make it ignore any space after
executing the begin block. The end block is a bit more tricky as special
processing occurs at the end of an environment. With the
\ci{ignorespacesafterend} \LaTeX{} will issue an \ci{ignorespaces} after the
special `end' processing has occurred.

\begin{example}
\newenvironment{simple}%
 {\noindent}%
 {\par\noindent}

\begin{simple}
See the space\\to the left.
\end{simple}
Same\\here.
\end{example}

\begin{example}
\newenvironment{correct}%
 {\noindent\ignorespaces}%
 {\par\noindent%
   \ignorespacesafterend}

\begin{correct}
No space\\to the left.
\end{correct}
Same\\here.
\end{example}

\subsection{Command-line \LaTeX}

If you work on a Unix-like OS, you might be using Makefiles to build your
\LaTeX{} projects. In that connection it might be interesting to produce
different versions of the same document by calling \LaTeX{} with command-line
parameters. If you add the following structure to your document:

\begin{verbatim}
\usepackage{ifthen}
\ifthenelse{\equal{\blackandwhite}{true}}{
  % "black and white" mode; do something..
}{
  % "color" mode; do something different..
}
\end{verbatim}

Now call \LaTeX{} like this:
\begin{verbatim}
latex '\newcommand{\blackandwhite}{true}\input{test.tex}'
\end{verbatim}

First the command \verb|\blackandwhite| gets defined and then the actual file is read with input.
By setting \verb|\blackandwhite| to false the color version of the document would be produced.

\subsection{Your Own Package}

If you define a lot of new environments and commands, the preamble of
your document will get quite long. In this situation, it is a good
idea to create a \LaTeX{} package containing all your command and
environment definitions. Use the \ci{usepackage}
command to make the package available in your document.

\begin{figure}[!htbp]
\begin{lined}{\textwidth}
\begin{verbatim}
% Demo Package by Tobias Oetiker
\ProvidesPackage{demopack}
\newcommand{\tnss}{The not so Short Introduction
                   to \LaTeXe}
\newcommand{\txsit}[1]{The \emph{#1} Short
                       Introduction to \LaTeXe}
\newenvironment{king}{\begin{quote}}{\end{quote}}
\end{verbatim}
\end{lined}
\caption{Example Package.} \label{package}
\end{figure}

Writing a package basically consists of copying the contents of
your document preamble into a separate file with a name ending in
\texttt{.sty}. There is one special command,
\begin{lscommand}
\ci{ProvidesPackage}\verb|{|\emph{package name}\verb|}|
\end{lscommand}
\noindent for use at the very beginning of your package
file. \verb|\ProvidesPackage| tells \LaTeX{} the name of the package
and will allow it to issue a sensible error message when you try to
include a package twice. Figure~\ref{package} shows a small example
package that contains the commands defined in the examples above.

\section{Fonts and Sizes}
\label{sec:fontsize}

\subsection{Font Changing Commands}
\index{font}\index{font size} \LaTeX{} chooses the appropriate font
and font size based on the logical structure of the document
(sections, footnotes, \ldots).  In some cases, one might like to change
fonts and sizes by hand. To do this, use the commands listed in
Tables~\ref{fonts} and~\ref{sizes}. The actual size of each font
is a design issue and depends on the document class and its options.
Table~\ref{tab:pointsizes} shows the absolute point size for these
commands as implemented in the standard document classes.

\begin{example}
{\small The small and
\textbf{bold} Romans ruled}
{\Large all of great big
\textit{Italy}.}
\end{example}

One important feature of \LaTeXe{} is that the font attributes are
independent. This means that issuing size or even font
changing commands, and still keep bold or slant attributes set
earlier.

In \emph{math mode} use the font changing \emph{commands} to
temporarily exit \emph{math mode} and enter some normal text. If you want to
switch to another font for math typesetting you need another
special set of commands; refer to Table~\ref{mathfonts}.

\begin{table}[!bp]
\caption{Fonts.} \label{fonts}
\begin{lined}{12cm}
%
% Alan suggested not to tell about the other form of the command
% e.g. \verb|\sffamily| or \verb|\bfseries|. This seems a good thing to me.
%
\begin{tabular}{@{}rl@{\qquad}rl@{}}
\fni{textrm}\verb|{...}|        &       \textrm{\wi{roman}}&
\fni{textsf}\verb|{...}|        &       \textsf{\wi{sans serif}}\\
\fni{texttt}\verb|{...}|        &       \texttt{typewriter}\\[6pt]
\fni{textmd}\verb|{...}|        &       \textmd{medium}&
\fni{textbf}\verb|{...}|        &       \textbf{\wi{bold face}}\\[6pt]
\fni{textup}\verb|{...}|        &       \textup{\wi{upright}}&
\fni{textit}\verb|{...}|        &       \textit{\wi{italic}}\\
\fni{textsl}\verb|{...}|        &       \textsl{\wi{slanted}}&
\fni{textsc}\verb|{...}|        &       \textsc{\wi{Small Caps}}\\[6pt]
\ci{emph}\verb|{...}|           &       \emph{emphasized} &
\fni{textnormal}\verb|{...}|    &       \textnormal{document} font
\end{tabular}

\bigskip
\end{lined}
\end{table}


\begin{table}[!bp]
\index{font size}
\caption{Font Sizes.} \label{sizes}
\begin{lined}{12cm}
\begin{tabular}{@{}ll}
\fni{tiny}      & \tiny        tiny font \\
\fni{scriptsize}   & \scriptsize  very small font\\
\fni{footnotesize} & \footnotesize  quite small font \\
\fni{small}        &  \small            small font \\
\fni{normalsize}   &  \normalsize  normal font \\
\fni{large}        &  \large       large font
\end{tabular}%
\qquad\begin{tabular}{ll@{}}
\fni{Large}        &  \Large       larger font \\[5pt]
\fni{LARGE}        &  \LARGE       very large font \\[5pt]
\fni{huge}         &  \huge        huge \\[5pt]
\fni{Huge}         &  \Huge        largest
\end{tabular}

\bigskip
\end{lined}
\end{table}

\begin{table}[!tbp]
\caption{Absolute Point Sizes in Standard Classes.}\label{tab:pointsizes}
\label{tab:sizes}
\begin{lined}{12cm}
\begin{tabular}{lrrr}
\multicolumn{1}{c}{size} &
\multicolumn{1}{c}{10pt (default) } &
           \multicolumn{1}{c}{11pt option}  &
           \multicolumn{1}{c}{12pt option}\\
\verb|\tiny|       & 5pt  & 6pt & 6pt\\
\verb|\scriptsize| & 7pt  & 8pt & 8pt\\
\verb|\footnotesize| & 8pt & 9pt & 10pt \\
\verb|\small|        & 9pt & 10pt & 11pt \\
\verb|\normalsize| & 10pt & 11pt & 12pt \\
\verb|\large|      & 12pt & 12pt & 14pt \\
\verb|\Large|      & 14pt & 14pt & 17pt \\
\verb|\LARGE|      & 17pt & 17pt & 20pt\\
\verb|\huge|       & 20pt & 20pt & 25pt\\
\verb|\Huge|       & 25pt & 25pt & 25pt\\
\end{tabular}

\bigskip
\end{lined}
\end{table}


\begin{table}[!bp]
\caption{Math Fonts.} \label{mathfonts}
\begin{lined}{0.7\textwidth}
\begin{tabular}{@{}ll@{}}
\fni{mathrm}\verb|{...}|&     $\mathrm{Roman\ Font}$\\
\fni{mathbf}\verb|{...}|&     $\mathbf{Boldface\ Font}$\\
\fni{mathsf}\verb|{...}|&     $\mathsf{Sans\ Serif\ Font}$\\
\fni{mathtt}\verb|{...}|&     $\mathtt{Typewriter\ Font}$\\
\fni{mathit}\verb|{...}|&     $\mathit{Italic\ Font}$\\
\fni{mathcal}\verb|{...}|&    $\mathcal{CALLIGRAPHIC\ FONT}$\\
\fni{mathnormal}\verb|{...}|& $\mathnormal{Normal\ Font}$\\
\end{tabular}

%\begin{tabular}{@{}lll@{}}
%\textit{Command}&\textit{Example}&    \textit{Output}\\[6pt]
%\fni{mathcal}\verb|{...}|&    \verb|$\mathcal{B}=c$|&     $\mathcal{B}=c$\\
%\fni{mathscr}\verb|{...}|&    \verb|$\mathscr{B}=c$|&     $\mathscr{B}=c$\\
%\fni{mathrm}\verb|{...}|&     \verb|$\mathrm{K}_2$|&      $\mathrm{K}_2$\\
%\fni{mathbf}\verb|{...}|&     \verb|$\sum x=\mathbf{v}$|& $\sum x=\mathbf{v}$\\
%\fni{mathsf}\verb|{...}|&     \verb|$\mathsf{G\times R}$|&        $\mathsf{G\times R}$\\
%\fni{mathtt}\verb|{...}|&     \verb|$\mathtt{L}(b,c)$|&   $\mathtt{L}(b,c)$\\
%\fni{mathnormal}\verb|{...}|& \verb|$\mathnormal{R_{19}}\neq R_{19}$|&
%$\mathnormal{R_{19}}\neq R_{19}$\\
%\fni{mathit}\verb|{...}|&     \verb|$\mathit{ffi}\neq ffi$|& $\mathit{ffi}\neq ffi$
%\end{tabular}

\bigskip
\end{lined}
\end{table}

In connection with the font size commands, \wi{curly braces} play a
significant role. They are used to build \emph{groups}.  Groups
limit the scope of most \LaTeX{} commands.\index{grouping}

\begin{example}
He likes {\LARGE large and
{\small small} letters}.
\end{example}

The font size commands also change the line spacing, but only if the
paragraph ends within the scope of the font size command. The closing curly
brace \verb|}| should therefore not come too early.  Note the position of
the \ci{par} command in the next two examples. \footnote{\texttt{\bs{}par}
is equivalent to a blank line}


\begin{example}
{\Large Don't read this!
 It is not true.
 You can believe me!\par}
\end{example}

\begin{example}
{\Large This is not true either.
But remember I am a liar.}\par
\end{example}

If you want to activate a size changing command for a whole paragraph
of text or even more, you might want to use the environment syntax for
font changing commands.

\begin{example}
\begin{Large}
This is not true.
But then again, what is these
days \ldots
\end{Large}
\end{example}

\noindent This will save you from counting lots of curly braces.

\subsection{Danger, Will Robinson, Danger}

As noted at the beginning of this chapter, it is dangerous to clutter
your document with explicit commands like this, because they work in
opposition to the basic idea of \LaTeX{}, which is to separate the
logical and visual markup of your document.  This means that if you
use the same font changing command in several places in order to
typeset a special kind of information, you should use
\verb|\newcommand| to define a ``logical wrapper command'' for the font
changing command.

\begin{example}
\newcommand{\oops}[1]{%
 \textbf{#1}}
Do not \oops{enter} this room,
it's occupied by \oops{machines}
of unknown origin and purpose.
\end{example}

This approach has the advantage that you can decide at some later
stage that you want to use a visual representation of danger other
than \verb|\textbf|, without having to wade through your document,
identifying all the occurrences of \verb|\textbf| and then figuring out
for each one whether it was used for pointing out danger or for some other
reason.

Please note the difference between telling \LaTeX{} to
\emph{emphasize} something and telling it to use a different
\emph{font}. The \ci{emph} command is context aware, while the font commands are absolute.

\begin{example}
\textit{You can also
  \emph{emphasize} text if
  it is set in italics,}
\textsf{in a
  \emph{sans-serif} font,}
\texttt{or in
  \emph{typewriter} style.}
\end{example}

\subsection{Advice}

To conclude this journey into the land of fonts and font sizes,
here is a little word of advice:\nopagebreak

\begin{quote}
  \underline{\textbf{Remember\Huge!}} \textit{The}
  \textsf{M\textbf{\LARGE O} \texttt{R}\textsl{E}} fonts \Huge you
  \tiny use \footnotesize \textbf{in} a \small \texttt{document},
  \large \textit{the} \normalsize more \textsc{readable} and
  \textsl{\textsf{beautiful} it bec\large o\Large m\LARGE e\huge s}.
\end{quote}

\section{Spacing}

\subsection{Line Spacing}

\index{line spacing} If you want to use larger inter-line spacing in a
document, change its value by putting the
\begin{lscommand}
\ci{linespread}\verb|{|\emph{factor}\verb|}|
\end{lscommand}
\noindent command into the preamble of your document.
Use \verb|\linespread{1.3}| for ``one and a half'' line
spacing, and \verb|\linespread{1.6}| for ``double'' line spacing.  Normally
the lines are not spread, so the default line spread factor
is~1.\index{double line spacing}

Note that the effect of the \ci{linespread} command is rather drastic and
not appropriate for published work. So if you have a good reason for
changing the line spacing you might want to use the command:
\begin{lscommand}
\verb|\setlength{\baselineskip}{1.5\baselineskip}|
\end{lscommand}

\begin{example}
{\setlength{\baselineskip}%
           {1.5\baselineskip}
This paragraph is typeset with
the baseline skip set to 1.5 of
what it was before. Note the par
command at the end of the
paragraph.\par}

This paragraph has a clear
purpose, it shows that after the
curly brace has been closed,
everything is back to normal.
\end{example}

\subsection{Paragraph Formatting}\label{parsp}

In \LaTeX{}, there are two parameters influencing paragraph layout.
By placing a definition like
\begin{code}
\ci{setlength}\verb|{|\ci{parindent}\verb|}{0pt}| \\
\verb|\setlength{|\ci{parskip}\verb|}{1ex plus 0.5ex minus 0.2ex}|
\end{code}
in the preamble of the input file, you can change the layout of
paragraphs. These two commands increase the space between two paragraphs
while setting the paragraph indent to zero.

The \texttt{plus} and \texttt{minus} parts of the length above tell
\TeX{} that it can compress and expand the inter-paragraph skip by the
amount specified, if this is necessary to properly fit the paragraphs
onto the page.

In continental Europe,
paragraphs are often separated by some space and not indented. But
beware, this also has its effect on the table of contents. Its lines
get spaced more loosely now as well. To avoid this, you might want to
move the two commands from the preamble into your document to some
place below the command \verb|\tableofcontents| or to not use them at all,
because you'll find that most professional books use indenting and not
spacing to separate paragraphs.


If you want to indent a paragraph that is not indented, use
\begin{lscommand}
\ci{indent}
\end{lscommand}
\noindent at the beginning of the paragraph.\footnote{To indent the first paragraph after each section head, use
  the \pai{indentfirst} package in the `tools' bundle.} Obviously,
this will only have an effect when \verb|\parindent| is not set to
zero.

To create a non-indented paragraph, use
\begin{lscommand}
\ci{noindent}
\end{lscommand}
\noindent as the first command of the paragraph. This might come in handy when
you start a document with body text and not with a sectioning command.

\subsection{Horizontal Space}

\label{sec:hspace}
\LaTeX{} determines the spaces between words and sentences
automatically. To add horizontal space, use: \index{horizontal!space}
\begin{lscommand}
\ci{hspace}\verb|{|\emph{length}\verb|}|
\end{lscommand}
If such a space should be kept even if it falls at the end or the
start of a line, use \verb|\hspace*| instead of \verb|\hspace|.  The
\emph{length} in the simplest case is just a number plus a unit.  The
most important units are listed in Table~\ref{units}.
\index{units}\index{dimensions}

\begin{example}
This\hspace{1.5cm}is a space
of 1.5 cm.
\end{example}
\suppressfloats
\begin{table}[tbp]
\caption{\TeX{} Units.} \label{units}\index{units}
\begin{lined}{9.5cm}
\begin{tabular}{@{}ll@{}}
\texttt{mm} & millimetre $\approx 1/25$~inch \quad \demowidth{1mm} \\
\texttt{cm} & centimetre = 10~mm  \quad \demowidth{1cm}                     \\
\texttt{in} & inch $=$ 25.4~mm \quad \demowidth{1in}                    \\
\texttt{pt} & point $\approx 1/72$~inch $\approx \frac{1}{3}$~mm  \quad\demowidth{1pt}\\
\texttt{em} & approx width of an `M' in the current font \quad \demowidth{1em}\\
\texttt{ex} & approx height of an `x' in the current font \quad \demowidth{1ex}
\end{tabular}

\bigskip
\end{lined}
\end{table}

\label{cmd:stretch}
The command
\begin{lscommand}
\ci{stretch}\verb|{|\emph{n}\verb|}|
\end{lscommand}
\noindent generates a special rubber space. It stretches until all the
remaining space on a line is filled up. If multiple
\verb|\hspace{\stretch{|\emph{n}\verb|}}| commands are issued on the same
line, they occupy all available space in proportion of their respective
stretch factors.


\begin{example}
x\hspace{\stretch{1}}
x\hspace{\stretch{3}}x
\end{example}

When using horizontal space together with text, it may make sense to make
the space adjust its size relative to the size of the current font.
This can be done by using the text-relative units \texttt{em} and
\texttt{ex}:

\begin{example}
{\Large{}big\hspace{1em}y}\\
{\tiny{}tin\hspace{1em}y}
\end{example}

\subsection{Vertical Space}
The space between paragraphs, sections, subsections, \ldots\ is
determined automatically by \LaTeX. If necessary, additional vertical
space \emph{between two paragraphs} can be added with the command:
\begin{lscommand}
\ci{vspace}\verb|{|\emph{length}\verb|}|
\end{lscommand}

This command should normally be used between two empty lines.  If the
space should be preserved at the top or at the bottom of a page, use
the starred version of the command, \verb|\vspace*|, instead of \verb|\vspace|.
\index{vertical space}

The \verb|\stretch| command, in connection with \verb|\pagebreak|, can
be used to typeset text on the last line of a page, or to centre text
vertically on a page.
\begin{code}
\begin{verbatim}
Some text \ldots

\vspace{\stretch{1}}
This goes onto the last line of the page.\pagebreak
\end{verbatim}
\end{code}

Additional space between two lines of \emph{the same} paragraph or
within a table is specified with the
\begin{lscommand}
\ci{\bs}\verb|[|\emph{length}\verb|]|
\end{lscommand}
\noindent command.

With \ci{bigskip} and \ci{smallskip} you can skip a predefined amount of
vertical space without having to worry about exact numbers.


\section{Page Layout}

\begin{figure}[!hp]
\begin{center}
\makeatletter\@mylayout\makeatother
\end{center}
\vspace*{1.8cm}
\caption[Layout parameters for this book.]{Layout parameters for this book. Try the \pai{layouts} package to print the layout of your own document.}
\label{fig:layout}
\cih{footskip}
\cih{headheight}
\cih{headsep}
\cih{marginparpush}
\cih{marginparsep}
\cih{marginparwidth}
\cih{oddsidemargin}
\cih{paperheight}
\cih{paperwidth}
\cih{textheight}
\cih{textwidth}
\cih{topmargin}
\end{figure}
\index{page layout}
\LaTeXe{} allows you to specify the \wi{paper size} in the
\verb|\documentclass| command. It then automatically picks the right
text \wi{margins}, but sometimes you may not be happy with
the predefined values. Naturally, you can change them.
%no idea why this is needed here ...
\thispagestyle{fancyplain}
Figure~\ref{fig:layout} shows all the parameters that can be changed.
The figure was produced with the \pai{layout} package from the tools bundle.%
\footnote{\CTANref|pkg/tools|}

\textbf{WAIT!} \ldots before you launch into a ``Let's make that
narrow page a bit wider'' frenzy, take a few seconds to think. As with
most things in \LaTeX, there is a good reason for the page layout to
be as it is.

Sure, compared to your off-the-shelf MS Word page, it looks awfully
narrow. But take a look at your favourite book\footnote{I mean a real
  printed book produced by a reputable publisher.} and count the number
of characters on a standard text line. You will find that there are no
more than about 66 characters on each line. Now do the same on your
\LaTeX{} page. You will find that there are also about 66 characters
per line.  Experience shows that the reading gets difficult as soon as
there are more characters on a single line. This is because it is
difficult for the eyes to move from the end of one line to the start of the next one.
This is also why newspapers are typeset in multiple columns.

So if you increase the width of your body text, keep in mind that you
are making life difficult for the readers of your paper. But enough
of the cautioning, I promised to tell you how you do it \ldots

\LaTeX{} provides two commands to change these parameters. They are
usually used in the document preamble.

The first command assigns a fixed value to any of the parameters:
\begin{lscommand}
\ci{setlength}\verb|{|\emph{parameter}\verb|}{|\emph{length}\verb|}|
\end{lscommand}

The second command adds a length to any of the parameters:
\begin{lscommand}
\ci{addtolength}\verb|{|\emph{parameter}\verb|}{|\emph{length}\verb|}|
\end{lscommand}

This second command is actually more useful than the \ci{setlength}
command, because it works relative to the existing settings.
To add one centimetre to the overall text width, I put the
following commands into the document preamble:
\begin{code}
\verb|\addtolength{\hoffset}{-0.5cm}|\\
\verb|\addtolength{\textwidth}{1cm}|
\end{code}

In this context, you might want to look at the \pai{calc} package.
It allows you to use arithmetic operations in the argument of \ci{setlength}
and other places where numeric values are entered into function
arguments.

\section{More Fun With Lengths}

Whenever possible, I avoid using absolute lengths in
\LaTeX{} documents. I rather try to base things on the width or height
of other page elements. For the width of a figure this could
be \verb|\textwidth| in order to make it fill the page.

The following 3 commands allow you to determine the width, height and
depth of a text string.

\begin{lscommand}
\ci{settoheight}\verb|{|\emph{variable}\verb|}{|\emph{text}\verb|}|\\
\ci{settodepth}\verb|{|\emph{variable}\verb|}{|\emph{text}\verb|}|\\
\ci{settowidth}\verb|{|\emph{variable}\verb|}{|\emph{text}\verb|}|
\end{lscommand}

\noindent The example below shows a possible application of these commands.

\begin{example}
\flushleft
\newenvironment{vardesc}[1]{%
  \settowidth{\parindent}{#1:\ }
  \makebox[0pt][r]{#1:\ }}{}

\begin{displaymath}
a^2+b^2=c^2
\end{displaymath}

\begin{vardesc}{Where}$a$,
$b$ -- are adjacent to the right
angle of a right-angled triangle.

$c$ -- is the hypotenuse of
the triangle and feels lonely.

$d$ -- finally does not show up
here at all. Isn't that puzzling?
\end{vardesc}
\end{example}

\section{Boxes}
\LaTeX{} builds up its pages by pushing around boxes. At first, each
letter is a little box, which is then glued to other letters to form
words. These are again glued to other words, but with special glue,
which is elastic so that a series of words can be squeezed or
stretched as to exactly fill a line on the page.

I admit, this is a very simplistic version of what really happens, but the
point is that \TeX{} operates on glue and boxes. Letters are not the only
things that can be boxes. You can put virtually everything into a box,
including other boxes. Each box will then be handled by \LaTeX{} as if it
were a single letter.

In earlier chapters you encountered some boxes, although I did
not tell you. The \ei{tabular} environment and the \ci{includegraphics}, for
example, both produce a box. This means that you can easily arrange two
tables or images side by side. You just have to make sure that their
combined width is not larger than the textwidth.

You can also pack a paragraph of your choice into a box with either
the

\begin{lscommand}
\ci{parbox}\verb|[|\emph{pos}\verb|]{|\emph{width}\verb|}{|\emph{text}\verb|}|
\end{lscommand}

\noindent command or the

\begin{lscommand}
\verb|\begin{|\ei{minipage}\verb|}[|\emph{pos}\verb|]{|\emph{width}\verb|}| text
\verb|\end{|\ei{minipage}\verb|}|
\end{lscommand}

\noindent environment. The \texttt{pos} parameter can take one of the letters
\texttt{c, t} or \texttt{b} to control the vertical alignment of the box,
relative to the baseline of the surrounding text. \texttt{width} takes
a length argument specifying the width of the box. The main difference
between a \ei{minipage} and a \ci{parbox} is that you cannot use all commands
and environments inside a \ei{parbox}, while almost anything is possible in
a \ei{minipage}.

While \ci{parbox} packs up a whole paragraph doing line breaking and
everything, there is also a class of boxing commands that operates
only on horizontally aligned material. We already know one of them;
it's called \ci{mbox}. It simply packs up a series of boxes into
another one, and can be used to prevent \LaTeX{} from breaking two
words. As boxes can be put inside boxes, these horizontal box packers
give you ultimate flexibility.

\begin{lscommand}
\ci{makebox}\verb|[|\emph{width}\verb|][|\emph{pos}\verb|]{|\emph{text}\verb|}|
\end{lscommand}

\noindent \texttt{width} defines the width of the resulting box as
seen from the outside.\footnote{This means it can be smaller than the
material inside the box. You can even set the
width to 0pt so that the text inside the box will be typeset without
influencing the surrounding boxes.}  Besides the length
expressions, you can also use \ci{width}, \ci{height}, \ci{depth}, and
\ci{totalheight} in the width parameter. They are set from values
obtained by measuring the typeset \emph{text}. The \emph{pos} parameter takes
a one letter value: \textbf{c}enter, flush\textbf{l}eft,
flush\textbf{r}ight, or \textbf{s}pread the text to fill the box.

The command \ci{framebox} works exactly the same as \ci{makebox}, but
it draws a box around the text.

The following example shows you some things you could do with
the \ci{makebox} and \ci{framebox} commands.

\begin{example}
\makebox[\textwidth]{%
    c e n t r a l}\par
\makebox[\textwidth][s]{%
    s p r e a d}\par
\framebox[1.1\width]{Guess I'm
    framed now!} \par
\framebox[0.8\width][r]{Bummer,
    I am too wide} \par
\framebox[1cm][l]{never
    mind, so am I}
Can you read this?
\end{example}

Now that we control the horizontal, the obvious next step is to go for
  the vertical.\footnote{Total control is only to be obtained by
  controlling both the horizontal and the vertical \ldots}
 No problem for \LaTeX{}. The


\begin{lscommand}
\ci{raisebox}\verb|{|\emph{lift}\verb|}[|\emph{extend-above-baseline}\verb|][|\emph{extend-below-baseline}\verb|]{|\emph{text}\verb|}|
\end{lscommand}

\noindent command lets you define the vertical properties of a
box. You can use \ci{width}, \ci{height}, \ci{depth}, and
  \ci{totalheight} in the first three parameters, in order to act
  upon the size of the box inside the \emph{text} argument.


\begin{example}
\raisebox{0pt}[0pt][0pt]{\Large%
\textbf{Aaaa\raisebox{-0.3ex}{a}%
\raisebox{-0.7ex}{aa}%
\raisebox{-1.2ex}{r}%
\raisebox{-2.2ex}{g}%
\raisebox{-4.5ex}{h}}}
she shouted, but not even the next
one in line noticed that something
terrible had happened to her.
\end{example}

\section{Rules}
\label{sec:rule}

A few pages back you may have noticed the command

\begin{lscommand}
\ci{rule}\verb|[|\emph{lift}\verb|]{|\emph{width}\verb|}{|\emph{height}\verb|}|
\end{lscommand}

\noindent In normal use it produces a simple black box.

\begin{example}
\rule{3mm}{.1pt}%
\rule[-1mm]{5mm}{1cm}%
\rule{3mm}{.1pt}%
\rule[1mm]{1cm}{5mm}%
\rule{3mm}{.1pt}
\end{example}

\noindent This is useful for drawing vertical and horizontal
lines. The line on the title page, for example, has been created with a
\ci{rule} command.

\bigskip
{\flushright The End.\par}

%

% Local Variables:
% TeX-master: "lshort2e"
% mode: latex
% mode: flyspell
% End:
